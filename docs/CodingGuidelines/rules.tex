\chapter{Coding Guidelines}
\label{ch_rules}

The software coding guidelines are segmented into three areas: the software version numbering system to be adopted, a \textit{style} guideline describing the general style that should be applied to all source code and \textit{coding rules} that must be adopted to promote software assurance.

\section{Version Numbering}

The following version numbering system will be employed:

\textbf{major.minor[.build[.revision]]}

\begin{itemize}
\item \textbf{major} --- Incremented when a major change occurs or the API is modified.
\item \textbf{minor} --- Incremented when a minor change in functionality occurs or a development milestone is reached. Reset to 0 when the major version number increments.
\item \textbf{build} --- Incremented every time the code is modified. Reset to 0 when the major or minor version number is increased.
\item \textbf{revision} --- Incremented when a previously released version is modified in some way to fix a bug or change its functionality.
\end{itemize}


\section{Style Guidelines}

\subsection{Enforced Rules}
A number of rules on coding style must be enforced in order to maintain the source code:

\begin{labeling}{100.}
\item [1.] \textbf{\textit{\#pragma once} should be used in favour of \textit{\#define} header guards.} \\
   Used to prevent multiple inclusion as \textit{\#define} header guards are prone to copy and paste errors. \\
\item [2.] \textbf{Include the defined legal statement with regards to software licensing in all source code files.} \\
   \begin{verbatim}
The MIT License (MIT)
Copyright (c) <year> <copyright holders>

Permission is hereby granted, free of charge, to any person obtaining a copy 
of this software and associated documentation files (the "Software"), to 
deal in the Software without restriction, including without limitation the 
rights to use, copy, modify, merge, publish, distribute, sublicense, and/or 
sell copies of the Software, and to permit persons to whom the Software is 
furnished to do so, subject to the following conditions:

The above copyright notice and this permission notice shall be included in 
all copies or substantial portions of the Software.

THE SOFTWARE IS PROVIDED "AS IS", WITHOUT WARRANTY OF ANY KIND, EXPRESS OR 
IMPLIED, INCLUDING BUT NOT LIMITED TO THE WARRANTIES OF MERCHANTABILITY, 
FITNESS FOR A PARTICULAR PURPOSE AND NONINFRINGEMENT. IN NO EVENT SHALL THE 
AUTHORS OR COPYRIGHT HOLDERS BE LIABLE FOR ANY CLAIM, DAMAGES OR OTHER 
LIABILITY, WHETHER IN AN ACTION OF CONTRACT, TORT OR OTHERWISE, ARISING 
FROM, OUT OF OR IN CONNECTION WITH THE SOFTWARE OR THE USE OR OTHER DEALINGS 
IN THE SOFTWARE.

\end{verbatim}
\item [3.] \textbf{The following copyright statement will be placed at the top of all source code files:}
\begin{verbatim}
/*****************************************************************************
 * Copyright (C) [INSTITUTION NAME], [YEAR]                                  *
 *                                                                           *
 * This file is part of libsafecrypto.                                       *
 *                                                                           *
 * This file is subject to the terms and conditions defined in the file      *
 * 'LICENSE', which is part of this source code package.                     *
 *****************************************************************************/
 \end{verbatim}
\item [4.] \textbf{Do not use git attributes or SVN keyword substitution.} \\
   This can lead to unnecessary file differences when working copies are updated, this in turn can lead to build systems performing unnecessary builds. \\
\item [5.] \textbf{To allow issues with source distributions to be tracked the distributed source code should contain tracking information that permits it to be related to the source code within the version control system. The following tagging information will be present in all source code files, when a source code distribution is built the tags will be modified by the build system with the relevant information.}
\begin{verbatim}
Git commit information:
  Author: $SC_AUTHOR$
  Date:   $SC_DATE$
  Branch: $SC_BRANCH$
  Id:     $SC_IDENT$
\end{verbatim}
\item [6.] \textbf{The limit on the length of lines is 80 columns.} \\
   Statements longer than 80 columns must be broken into sensible chunks, unless exceeding 80 columns significantly increases readability and does not hide information. Descendants are always substantially shorter than the parent and  are placed substantially to the right. The same applies to function headers with a long argument list. Never break user-visible strings, however, because that breaks the ability to grep for them. \\
\item [7.] \textbf{Use the defined SAFEcrypto data types} \\
   This will ease porting to different processor architectures. \\
\item [8.] \textbf{Always use the provided macros for dynamic memory allocation} \\
   These macros ensure that secure memory allocation is always undertaken, and can be easily modified when required to do so on a particular system. \\
\item [9.] \textbf{Use the provided debug macros} \\
   This is intended to allow development to place debug print statements in the
   source code that can be removed when a release library is built. \\
\item [10.] \textbf{Conditional statements should place a constant on the left} \\
   This is to prevent programming errors where assignments are implemented where the intention is to implement a test expression.
\end{labeling}


\subsection{Automated Rules}
Coding styles such as alignment, brackets and spacing can be manipulated by a CI script (using source code formatters) to ensure that source code committed to the revision system adheres to the style guidelines.

The following coding style is advised (adapted from the \href{https://www.openssl.org/policies/codingstyle.html}{OpenSSL Coding Style}) and will be enforced by automated tools when building a source distribution:

\begin{labeling}{100.}
\item [1.] \textbf{Use an adapted \textit{Kernighan and Ritchie} indent style.}
  
   \hspace{1cm} Functions are a special case where the opening bracket is at the start of the next line:
   \begin{verbatim}
        int function(int x)
        {
            body of function
        }
   \end{verbatim}
   \hspace{1cm} \textnormal{Condition statements:}
   \begin{verbatim}
        if (x_is_true) {
            do_y();
            do_z();
        }
   \end{verbatim}
   \hspace{1cm} \textnormal{Switch statements:}
   \begin{verbatim}
        switch (suffix) {
        case 'G':
        case 'g':
            mem <<= 30;
            break;
        case 'M':
        case 'm':
            mem <<= 20;
            break;
        case 'K':
        case 'k':
            mem <<= 10;
            /* fall through */
        default:
            break;
        }
   \end{verbatim}
   \hspace{1cm} Closing braces are not on an empty line of their own when followed by a continuation of the statement, but \textit{cuddled} braces will not be used:
   \begin{verbatim}
        do {
            ...
        } while (condition);

        if (x == y) {
            ...
        }
        else if (x > y) {
            ...
        }
        else {
            ...
        }
   \end{verbatim}
   \hspace{1cm} Always use braces around a single statement to help prevent copy and paste errors:
   \begin{verbatim}
        if (condition) {
            action();
        }
    
        if (condition) {
            do_this();
        }
        else {
            do_that();
        }
   \end{verbatim}
\item [2.] \textbf{Names with leading and trailing underscores should NOT be used.} \\
\item [3.] \textbf{\textit{\#define} constants and macros should be CAPITALISED.}
   \begin{verbatim}
        #define PARAM_THRESHOLD   5
   \end{verbatim}
\item [4.] \textbf{Enum constants should be CAPITALISED.}\\
   \begin{verbatim}
        typedef enum foobar {
            SC_ENUM_FOO = 0,
            SC_ENUM_BAR,
        } enum foobar\_e;
   \end{verbatim}
\item [5.] \textbf{Function, typedef, arguments and variable names, as well as struct, union and enum tag names should use underscore\_case.}\\
   \begin{verbatim}
        void function\_name(char *data_string, size_t len\_string);

        int variable\_name = 0;

        typedef struct my\_struct\_name my\_struct\_t;
   \end{verbatim}
\item [6.] \textbf{Global variables should use the prefix ``g\_''.}\\
\item [7.] \textbf{Pointer names should be positioned next to the ``*'' not the type name.}
   \begin{verbatim}
        int *pointer\_name;
   \end{verbatim}
\item [8.] \textbf{Use a space after C language keywords.}\\
\item [9.] \textbf{Do not add spaces around the inside of parenthesized expressions.}\\
\item [10.] \textbf{Use one space on either side of binary and ternary operators.}\\
\item [11.] \textbf{Do not put a space after unary operators.}\\
\item [12.] \textbf{Do not leave trailing whitespace at the ends of lines.}\\
\item [13.] \textbf{Separate functions with one blank line in the source file.}\\
\item [14.] \textbf{Always include parameter names with their data types in function prototypes.}\\
\item [15.] \textbf{In functions with complex return conditions and cleanup use the \textit{goto} statement as a mechanism to improve readability, reduce the possibility of errors and reduce excessive control structures.}
\item [16.] \textbf{To be continued\ldots}
\end{labeling}

\subsection{Doxygen Style}

Doxygen does not require any additional comments or special syntax added to source code in order to create reference documentation in the selected HTML, man pages or PDF.\@ However, developers can increase the usefulness of the reference documentation by adding comments and explanations directly to the source code using the required syntax.

Doxygen is very flexible in terms of the style that can be used to add documentation, however the following Doxygen style is encouraged for conformity. Developers can utilise more elaborate features of Doxygen if they desire to do so, e.g.\ describing function parameters and return values, grouping source files for documentation, etc.

\begin{labeling}{10.}
\item [1.] \textbf{Brief comment before}

Add an extra ``/''

\begin{verbatim}
    /// This method does something
    void DoSomething();
\end{verbatim}


\item [2.] \textbf{Brief comment after}

Add an extra ``/<''

\begin{verbatim}
    void DoSomething(); ///< This method does something
\end{verbatim}


\item [3.] \textbf{Detailed comment before}

Add an extra ``*'' [the intermediate leading ``*''s are optional]

\begin{verbatim}
    /** This is a method that does so much that I must
      * write an epic novel just to describe how much
      * it truly does. */
    void DoNothing();
\end{verbatim}


\item [4.] \textbf{Detailed comment after}

Add an extra ``*<'' [the intermediate leading ``*''s are optional]

\begin{verbatim}
    void DoNothing(); /**< This is a method that does so much
      * that I must write an epic novel just to describe how
      * much it truly does. */
\end{verbatim}

\end{labeling}


\newpage
\section{Coding Rules}

The following coding guidelines will be enforced using code reviews and automated rule checking with lint tools. The rules are segmented into three groups for clarity. Those rules that are not considered security specific, but are generic in nature and therefore applicable to the reliability and robustness of all software are given special consideration.

These rules have been derived from the SEI CERT C Coding Standard~\cite{sei_cert}. Developers are encouraged to familiarise themselves with this coding standard and use it as a reference when interpreting the rules given in this decoument. Other coding standards such as MISRA~\cite{misra_c} are also a useful resource.

\textbf{NOTE:} It is envisaged that to permit a pragmatic approach to development a \textit{debug} build of the package will be permitted for internal use that generates warnings rather than errors with regards to broken coding rules.

\subsection{Critical Security Rules}

\begin{labeling}{C.1000}
\item [C.1] Guarantee that library functions do not form invalid pointers [ARR38-C].
\item [C.2] Guarantee that storage for strings has sufficient space for character data and the null terminator [STR31-C].
\item [C.3] Do not pass a non-null-terminated character sequence to a library function that expects a string [STR32-C].
\item [C.4] Do not confuse narrow and wide character strings and functions [STR38-C].
\item [C.5] Exclude user input from format strings [FIO30-C].
\item [C.6] Distinguish between characters read from a file and \textit{EOF} or \textit{WEOF} [FIO34-C].
\item [C.7] Do not assume that \textit{fgets~()} or \textit{fgetws~()} returns a nonempty string when successful [FIO37-C].
\item [C.8] Do not call \textit{system~()} [ENV33-C].
\item [C.9] Call only asynchronous-safe functions within signal handlers [SIG30-C].
\item [C.10] Detect and handle standard library errors [ERR33-C].
\item [C.11] Properly seed pseudorandom number generators [MSC32-C].
\item [C.12] Do not pass invalid data to the \textit{asctime~()} function [MSC33-C].
\end{labeling}


\subsection{Important Security Rules}

\begin{labeling}{I.1000}
\item [I.1] Declare objects with appropriate storage durations [DCL30-C].
\item [I.2] Do not declare an identifier with conflicting linkage classifications [DCL36-C].
\item [I.3] Do not depend on the order of evaluation for side effects [EXP30-C].
\item [I.4] Do not access a volatile object through a nonvolatile reference [EXP32-C].
\item [I.5] Do not perform assignments in selection statements [EXP45-C].
\item [I.6] Ensure that unsigned integer operations do not wrap [INT30-C].
\item [I.7] Ensure that integer conversions do not result in lost or misinterpreted data [INT31-C].
\item [I.8] Ensure that operations on signed integers do not result in overflow [INT32-C].
\item [I.9] Ensure that division and remainder operations do not result in divide-by-zero errors [INT33-C].
\item [I.10] Do not use floating-point variables as loop counters [FLP30-C].
\item [I.11] Prevent or detect domain and range errors in math functions [FLP32-C].
\item [I.12] Do not attempt to modify string literals [STR30-C].
\item [I.13] Cast characters to unsigned char before converting to larger integer sizes [STR34-C].
\item [I.14] Use valid format strings [FIO47-C].
\item [I.15] Prevent data races when accessing bit-fields from multiple threads [CON32-C].
\item [I.16] Do not call \textit{signal~()} in a multithreaded program [CON37-C].
\item [I.17] Do not use the \textit{rand~()} function for generating pseudorandom numbers [MSC30-C].
\end{labeling}


\subsection{Minor Security Rules}

\begin{labeling}{M.1000}
\item [M.1] Avoid side effects in arguments to unsafe macros [PRE31-C].
\item [M.2] Do not use preprocessor directives in invocations of function-like macros [PRE32-C].
\item [M.3] Declare identifiers before using them [DCL31-C].
\item [M.4] Use the correct syntax when declaring a flexible array member [DCL38-C].
\item [M.5] Avoid information leakage when passing a structure across a trust boundary [DCL39-C].
\item [M.6] Do not modify objects with temporary lifetime [EXP35-C].
\item [M.7] Do not cast pointers into more strictly aligned pointer types [EXP36-C].
\item [M.8] Call functions with the correct number and type of arguments [EXP37-C].
\item [M.9] Do not modify constant objects [EXP40-C].
\item [M.10] Avoid undefined behavior when using restrict-qualified pointers [EXP43-C].
\item [M.11] Do not shift an expression by a negative number of bits or by greater than or equal to the number of bits that exist in the operand. [INT34-C]
\item [M.12] Use correct integer precisions [INT35-C].
\item [M.13] Always use \textit{intptr\_t} and \textit{uintptr\_t} to convert between integers and pointers [INT36-C].
\item [M.14] A pointer should store a pointer only, high order bits should not be used to store flags [INT36-C].
\item [M.15] Ensure that floating-point conversions are within range of the new type [FLP34-C].
\item [M.16] Preserve precision when converting integral values to floating-point type [FLP36-C].
\item [M.17] Do not use object representations to compare floating-point values [FLP37-C].
\item [M.18] Arguments to character-handling functions must be representable as an unsigned char [STR37-C].
\item [M.19] Allocate and copy structures containing a flexible array member dynamically [MEM32-C].
\item [M.20] Do not modify the alignment of objects by calling \textit{realloc~()} [MEM36-C].
\item [M.21] Do not return from a computational exception signal handler [SIG35-C].
\item [M.22] Do not rely on indeterminate values of errno [ERR32-C].
\end{labeling}


\subsection{Generic Rules}

\begin{labeling}{G.1000}
\item [G.1] All exit handlers must return normally [ENV32-C].
\item [G.2] Do not form or use out-of-bounds pointers or array subscripts [ARR30-C].
\item [G.3] Ensure size arguments for variable length arrays are in a valid range [ARR32-C].
\item [G.4] Free dynamically allocated memory when no longer needed [MEM31-C].
\item [G.5] Allocate sufficient memory for an object [MEM35-C].
\item [G.6] Do not alternately input and output from a stream without an intervening flush or positioning call [FIO39-C].
\item [G.7] Avoid TOCTOU (Time-of-Check to Time-of-Use) race conditions while accessing files [FIO45-C].
\item [G.8] Set errno to zero before calling a library function known to set errno, and check errno only after the function returns a value indicating failure [ERR30-C].
\item [G.9] Do not refer to an atomic variable twice in an expression [CON40-C].
\item [G.10] Ensure that control never reaches the end of a non-void function [MSC37-C].
\item [G.11] Do not declare or define a reserved identifier [DCL37-C].
\item [G.12] Reset strings on \textit{fgets~()} or \textit{fgetws~()} failure [FIO40-C].
\item [G.13] Do not call \textit{getc~(), putc~(), getwc~(),} or \textit{putwc~()} with a stream argument that has side effects [FIO41-C].
\item [G.14] Close files when they are no longer needed [FIO42-C].
\item [G.15] Only use values for \textit{fsetpos~()} that are returned from \textit{fgetpos~()} [FIO44-C].
\item [G.16] Do not modify the object referenced by the return value of certain functions [\textit{getenv~(), setlocale~(), locale-conv~(), asctime~(), or strerror~()}] [ENV30-C].
\item [G.17] Do not rely on an environment pointer following an operation that may invalidate it [ENV31-C].
\item [G.18] Do not store pointers returned by certain functions [\textit{getenv~(), asctime~(), localeconv~(), setlocale~(), and strerror~()}] [ENV34-C].
\item [G.19] Do not call \textit{signal~()} from within interruptible signal handlers.
\item [G.20] Clean up thread-specific storage [CON30-C].
\item [G.21] Use threadsafe C library functions in multithreaded code [CON33-C].
\item [G.22] Declare objects shared between threads with appropriate storage durations [CON34-C].
\item [G.23] Do not treat a predefined identifier as an object if it might only be implemented as a macro [\textit{assert, errno, math\_errhandling, setjmp, va\_arg, va\_copy, va\_end,} and \textit{va\_start}] [MSC38-C].
\end{labeling}
